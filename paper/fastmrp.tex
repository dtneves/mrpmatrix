\documentclass[a4paper,10pt]{article}
\usepackage[utf8x]{inputenc}
\usepackage{algorithmic}
\usepackage{amsmath,amsfonts,amsthm,amssymb}

%opening
\title{}
\author{}

\begin{document}

\maketitle FastMRP

\section{Methods}
\subsection{Algorithmic Ideas}
To be able to compute MRP matrices for the largest datasets available today
(tens of thousands of sequences), memory and running time need to be optimized
simultaneously. To this end, we have used two main ideas in our algorithm.

On the larger datasets available today, an MRP matrix can contain hundreds
of thousands of columns and tens of thousands of rows, and these numbers will
likely grow dramatically in the future. Therefore, constructing the whole MRP
matrix in memory and then writing it to a file is not feasible. Neither is
keeping a larger number of question marks in memory an efficient approach. The
simple insight that MRP matrices are typically
very sparse can help reducing the memory footprint dramatically. A memory
efficient approach can be devised to keep in memory two sets of more compact
information: i) lists of bi-partitions induced by different trees, ii) the list
of sequences included in each tree. These two sets of information are sufficient
to create the MRP matrix. Furthermore, for each bi-partition, one can save
only the sequences on one side of the partition, and the rest of sequences
included in that tree should be on the other side. In addition, since writing to
the final file happens row by row, it is more efficient to keep the above
information on a per sequence basis. That is, we need a data-structure that, for
each sequence, can efficiently return which trees the sequence belongs to,
what bi-partitions are induced for each of those trees, and whether the sequence
is on the 1 side or the 0 side of each of those bi-partitions. Given such a data
structure, outputting the MRP matrix to a file is easy and efficient.

To create the aforementioned data structure, one can create the trees in memory
first and then traverse them in some fashion to extract the list of
bi-partitions. However, whenever creating the MRP matrix is the only goal of a
program, a more efficient approach is to build the list of bi-partitions on the
fly as the trees are being read from the input. If the trees are given in the
Newick format, the bi-partitions can be induced by stacking the open brackets
and sequences following them and then matching the open brackets on top
of the stack against the close brackets. The details of how these two
ideas can be implemented are given next. 

\subsection{Algorithmic Design}
Let's assume we are given a set of trees $\mathcal{T}=\{T_0
\ldots T_n\}$ in the Newick format on a disjoint but potentially overlapping set
of taxa. For example, we can have $\mathcal{T} =
\{T_0=(A,B,(C,D));,T_1=(C,(A,(E,F)),D);\}$. The goal is to create an MRP
matrix. For the given example, a potential MRP matrix is:\\

\noindent
$A~0~1~0$\\
$B~0~?~?$\\
$C~1~0~0$\\
$D~1~0~0$\\
$E~?~1~1$\\
$F~?~1~1$\\

Let's assume we have a function $m(s)$ that maps a sequence $s$
to a sequence index. Creating such a function is trivial. In the above example,
we can index the sequences as we encounter them so that our trees can be
represented as $\{T_0=[(0,1,(2,3));],T_1=[(2,(0,(4,5)),3);]\}$.

Our algorithm makes use of the following data structures to keep the
bi-partitions in the memory:
\begin{description}
 \item[$TreesPerSeq$] is an indexed list of sets of tree indices. Each
element in $TreesPerSeq$ is itself a set that corresponds to a particular
sequence. Each set contains indices of trees that the sequence belong to. In
the given example, $TreesPerSeq[0]$ is
$\{0,1\}$ to indicate that sequence $0$ appears in $T_0$ and $T_1$, while
$TreesPerSeq[4] = \{1\}$ to indicate that $4$ appears only in $T_1$.
 \item[$ColumnsPerSeq$] is an indexed list of lists of column indices. Each
element in $ColumnsPerSeq$ corresponds to a sequence and is itself a list of
column indices in the MRP where that sequence should have a '1'. In the given
example, $ColumnsPerSeq[0]=\{1\}$ to indicate that sequence 0 has a ``1'' only
in the second column of the MRP matrix. Or, $ColumnsPerSeq[4]=\{1,2\}$ to
indicate sequence $4$ has a ``1'' in the second and third columns.
\item[$EndIndexPerTree$] is a list of column indices. This list contains for
each tree, the index of the last column in the MRP matrix that relates to a
bi-partition in that tree. In our example, $T_0$ has only one bipartition,
appearing in the column $0$ of the MRP matrix, and therefore
$EndIndexPerTree[0]=0$. Tree $T_1$ has two bi-partitions and the last one
appears in column $2$. Therefore, $EndIndexPerTree[1]=2$.
 \end{description} 

$ColumnsPerSeq$ tells us where to put a ``$1$'' for each sequence. The two
other lists provide the information necessary to decide between a ``$0$'' and a
``$?$''. From $EndIndexPerTree$ we can find the tree corresponding to each
column. If the tree corresponding to a column does not contain a sequence (based
on $TreesPerSeq$) it has to be a ``$?$''. Otherwise, it is a ``$0$''.
Our algorithm to create an MRP matrix given these three data-structures is
outlined in Figure~\ref{fig:alg-out}.

\begin{figure}[htp]
 \centering
\begin{algorithmic}[5]
\FOR{each sequence index $s$}
	\STATE output the name of the sequence in a new row
	\STATE Let $c=0$
	\STATE Let $oneColumns$ be the $s^{th}$ element of $ColumnsPerSeq$
	\STATE Let $nextOne$ be the first element of $oneColumns$.
	\FOR{each tree index $t$}
	  \STATE let $tEnd$ be $t^{th}$ element of $EndIndexPerTree$
	  \IF {$TreesPerSeq[s]$ contains $t$}
	    \WHILE{$c \leq tEnd$}
		\IF {$nextOne$ is $c$}
		  \STATE write a $1$ to the output
		  \STATE let $nextOne$ be the next element of $oneColumns$.
		\ELSE
		  \STATE write a $0$ to the output
		\ENDIF
		\STATE  Increment $c$
	    \ENDWHILE
	  \ELSE
	    \STATE write $?$ to output $tEnd - c + 1$ times
	    \STATE Increase $c$ to $tEnd+1$
	  \ENDIF
	\ENDFOR
\ENDFOR
\end{algorithmic}
\caption{Outputting MRP}
 \label{fig:alg-out}
\end{figure}

To create the data-structures outlined above, we use a stack
data-structure:
\begin{description}
 \item[$S$] is an stack of lists of sequence indices (i.e, each element in the
stack is itself a list). Each list, when completed, corresponds to a
bipartition, which in turn corresponds to a column in the final MRP matrix.
 \end{description}
The above data-structure is used as follows. Each tree $T_i$ given in the Newick
format is read from the input. The Newick representation of the tree can
be trivially broken up into a list of tokens (using regular expressions for
example). Such a list is obtained and the tokens are traversed in order. 
A new empty list is pushed to the stack whenever a ``('' is encountered (the
first open bracket is skipped). When sequence names are encountered, they are
added to the list on top of the stack (if it exists). In addition, $TreesPerSeq$
is updated to indicate that the observed sequence is present in the current
tree. When a ``)'' is reached, the elements in the list on top of the stack form
a bipartition. The ``top'' element is popped out of the stack. Then, a new
column in the matrix is created by updating the $ColumnsPerSeq$ of each sequence
in the ``top'' list with an incrementing column index. If the stack is not
empty, the sequences in the top list also belong to another bi-partition. They
are therefore added to the list that is now on top of the stack. This process is
repeated until we reach a semi-colon, indicating the end of the tree. At this
point, $EndIndexPerTree$ is updated with the incrementing column index to mark
where the tree ends. This process is described in the algorithm outlined in
Figure~\ref{fig:alg-read}.

\begin{figure}[ht]
 \centering
\begin{algorithmic}[5]
\STATE Let $c=-1$
\FOR{each Newick tree $T$ indexed $i$}
	\FOR{each token $tok$ in $T$}
		\IF {$tok$ is a '('}
		      \STATE Push a new empty list to $S$
		\ENDIF
		\IF {$tok$ is a sequence name and $S$ is not empty}
		      \STATE Add $m(tok)$ to the list on top of $S$
		      \STATE Add $i$ to the list on the $m(tok)^{th}$ element of
$TreePerSeq$
		\ENDIF
		\IF {$tok$ is a ')' and $S$ is not empty}
		      \STATE Pop $top$ from $S$
		      \STATE  Increment $c$
		      \FOR{each sequence index $s$ is $top$}
			\STATE Add $c$ to the list on the $s^{the}$ position
of $ColumnsPerSeq$
		      \ENDFOR
		      \IF {$S$ is not empty}
			  \STATE  Add all elements of $top$ to the list on top
of $S$
		      \ENDIF
		\ENDIF
		\IF {$tok$ is a ';'}
		      \STATE Add $c$ to $EndIndexPerTree$
		\ENDIF
	\ENDFOR
\ENDFOR
\end{algorithmic}
 \caption{Reading Input and Creating the Data-structures}
 \label{fig:alg-read}
\end{figure}

As an example, consider the second tree in our example above, that is
$T_1=[(2,(0,(4,5)),3);$. Let's assume from the previous tree, we have kept a
current column index ($c$), which is now at 0 (since the first tree has only
one column). Initially $S$ is empty. The first open bracket is skipped.
When sequence $2$ is read, it is not added to $S$ since the stack is empty
initially. However, the set at $TreePerSeq[2]$ is updated to include the index
of current tree (i.e. 1). After reading the second ``('', we add an empty list
(let's call it $l_1$) to $S$. Then sequence $0$ is read and is added to $l_1$
and $TreePerSeq[0]$ is updated with a 1. Then, we encounter another ``('' and
therefore add a second list to stack (call it $l_2$). Then $4$ and $5$ are read
and are added to $l_2$, in addition to updating $TreePerSeq[4]$
and $TreePerSeq[5]$ with a 1. Then we encounter a ``)''. The contents of $l_2$
constitute a bipartition. We, therefore, increment $c$ to 1 and then add $c=1$
to $ColumnsPerSeq[4]$ and $ColumnsPerSeq[5]$ to indicate that sequences $4$ and
$5$ have a character ``1'' in column 1 of their MRP. Next, $l_2$ is removed and
all its elements are added to $l_1$ so that it contains $0,4,5$. Then another )
is read, and the above process repeats. This time $c$ is incremented to 2 and a
2 is added to $ColumnsPerSeq[0]$, $ColumnsPerSeq[4]$ and
$ColumnsPerSeq[5]$. Since after this step the stack becomes empty, reading $3$
only updates $TreePerSeq[3]$ and does not affect $S$ or $ColumnsPerSeq$. After
the semi-colon is read, $EndIndexPerTree[1]$ (where 1 is the tree index) is
updated to current value of c, which is 2. 
\end{document}

